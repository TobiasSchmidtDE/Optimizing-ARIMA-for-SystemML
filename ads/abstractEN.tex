%!TEX root = ../dokumentation.tex

\pagestyle{plain}

\iflang{en}{%
\addchap{Abstract - English} % Text für Überschrift

The \acl{ARIMA} (\acs{ARIMA}) model is one of most broadly used time series models and applied to better understand and forecast time series data. As a generalization of the \acl{ARMA} (\acs{ARMA} model, which itself is a combination of the \acl{AR} and \acs{MA} model, it can also be used to analyze time series that are non-stationary.

The goal of this research project was to complete the implementation of the \acs{ARIMA} algorithm for SystemML using the \acl{DML}.

This report starts of with a short introduction that gives an example for the applications of large scale time series analysis to demonstrate its importance and then discusses the focus and motivation of the project in more detail. In the second chapter, the \acs{ARIMA} model is explained and discussed in full with a heavy focus on the systems of linear equations which is at the center of the model. An important part of the algorithm is to find a solution for these linear systems, which is why there is a completely separate section in the theory chapter that discusses different approaches used for solving linear systems and particular the linear systems produced by \acs{ARIMA} models, which have some key characteristics that can be exploited to find a solution more quickly. To give some more background on the over all training algorithm a number of optimization methods commonly used are outlined and discussed. The third chapter then explains why and which solvers for the system of linear equations were chosen and discusses the implementation of \acs{ARIMA} in \acs{DML}. It is also describing how exactly the performance and precision of the script will be measured and how it is tested for correctness. Afterwards, the results are presented and finally the methodology as well as the results are discussed and assessed.

}